\section{Architettura}
	Il software è stato pensato per esser versatile a futuri sviluppi anche commerciali, focalizzando l'attenzione sulla
	\emph{modularità} e sull'\emph{estensibilità} del prodotto. In una concezione verticale (\emph{stack architetturale})
	della struttura, a basso livello si trova \emph{alg}, cioè il risolutore metaeuristico implementato e funzione \emph{core}
	del progetto, mentre ad alto livello si trovano le interfacce con l'utente, delle quali per ora è implementata solo \emph{cli}
	(command-line-interface). La \emph{cli} permette un'interazione descritta accuratamente nella sezione ``Casi d'Uso''.
	Tra questi due macrolivelli vi è un \emph{socket hard-soft} che gestisca l'interazione alto-basso in modo trasparente per i 
	moduli realmente implementati. Esso inoltre contiene il layer \emph{stats} per gestire le statistiche, dandole diversa forma
	a seconda del caso d'uso richiesto.
	\begin{figure}[H] 
	 	\begin{center}\includegraphics[width=12cm]{./images/stack-arch.pdf}
	 	\end{center} \caption{Architettura del software progettata
	e in parte implementata.} \label{fig:stackArch}
 	\end{figure}
	\subsection{Estensioni previste}
	\subsubsection{Interfacce grafiche}
	Tramite il software \emph{JNI} (Java Native Interface) è possibile collegarsi a tempo di compilazione con programmi Java.
	È pensabile quindi una \emph{gui} per l'interazione, o ancora più in là un applicativo Android (al limite un mero client)
	per demandare al cuore del programma la risoluzione di problemi di algoritmi di ottimizzazione.
	Infine è stato teorizzato un livello grafico Matlab per la presentazione di grafici statistici per l'utente e di evoluzione
	dell'algoritmo per i designer. 
	\subsubsection{Risolutori}
	Volendo in futuro testare ulteriori risolutori metaeuristici, approssimati o esatti senza per questo modificare la prima versione 
	del risolutore implementata, dev'essere possibile collegare ``parallelamente'' ad essa le nuove versioni. In questo senso il socket
	intermedio agisce da \emph{multiplexer} sul livello basso.

\section{Parametri e configurazione}
\label{sec:cfg}
	Automatizzare l'inserimento del problema nel programma richiede una certa complessità di progettazione che però viene 
	ampiamente ripagata in termini di tempo e usabilità.
	Un parser permette la piena personalizzazione del programma in un'ottica di versione Release, 
	dove al customer non è generalmente concesso di ricompilare
	frequentemente valori che nativamente è meglio inserire tramite un \emph{file di configurazione}. 
	I parametri di programma specificati dal file sono i seguenti:
	\begin{itemize}
	  \item "percorso grafo": \emph{stringa}
	  \item "numero veicoli": \emph{intero}
	  \item "nome immagine costruttiva": \emph{stringa}
	  \item "bmp per LOS \{true,false\}": \emph{stringa}
	  \item "octave per LOS \{true,false\}": \emph{stringa}
	  \item "metodo di valutazione \{0,1,2\} (pure,smart,wise)": \emph{intero}
	  \item "iterazioni massime LOS": \emph{intero}
	  \item "iterazioni peggiorative massime": \emph{intero}
	  \item "iterazioni massime STM": \emph{intero}
	  \item "tenure base ADD": \emph{intero}
	  \item "tenure base REMOVE": \emph{intero}
	\end{itemize}

\section{Casi d'uso}
%% CASO USO A - ANALISI SINGOLA ISTANZA
\begin{usecase}
	\addtitle{Caso d'uso A}{Analisi di singola istanza} 
	\addfield{Scope:}{System-wide} %the system under design
	%Level: "user-goal" or "subfunction"
	\addfield{Goal:}{ottenere una risoluzione metaeuristica di un certo problema UCARPP.}
	\addfield{Attore primario:}{l'utente finale}
	%Stakeholders and Interests: Who cares about this use case and what do they want?
	%\additemizedfield{Stakeholders and Interests:}{
	%	\item Stakeholder 1 name: his interests
	%}
	\addfield{Precondizioni:}{
		\begin{itemize}
		  \item un file locale nel quale sia correttamente espressa la configurazione
		  di un problema.
		\end{itemize}}
	\addfield{Postcondizioni:}{un report corretto \footnote{una struttura dati contenente:
		\begin{itemize}
		  \item profitto totale
		  \item costo totale, come somma dei tempi utilizzati dai veicoli
		  \item lista di $K$ \emph{percorsi}, ognuno descritto come:
		  \begin{itemize}
		     \item lista dei lati percorsi, nella forma \; $(v_1,\,v_i)\; (v_i,\,v_j)\; (v_i,\,v_j)\; \ldots \; (v_m,\,v_1)$
		     \item lista dei lati serviti, quindi un sottoinsieme della lista precedente, nel medesimo formato
		     \item profitto del percorso
		     \item costo temporale del percorso
		     \item domanda totale servita nel percorso
		     \end{itemize} 
		\end{itemize}},
		il cui formato è adattato a quello utilizzato nelle benchmark.}
	
	\addscenario{Scenario principale di successo:}{
		\item l'utente chiama il programma indicando quale problema vuole risolvere con quanti veicoli.
		\item il programma effettua la sua analisi e restituisce un report testuale di quanto trovato.
	}
	\addscenario{Estensioni:}{
		\item[1.a] la sinopsi è invalida:
			\begin{enumerate}
			\item[1.] il sistema informa in modo generale dell'errore
			\item[2.] il programma termina senza successo.
			\end{enumerate}
		\item[1.b] il file indicato non è presente o è in un formato non valido:
			\begin{enumerate}
			\item[1.] il sistema informa del file non trovato
			\item[2.] il programma termina senza successo.
			\end{enumerate}
		\item[2.b] è implementato il modulo Matlab/Octave:
			\begin{enumerate}
			\item[1.] viene scritto uno script per graficare la serie storica della
			funzione obiettivo e della selezione delle mosse.
			\end{enumerate}
	}
\end{usecase}

%% CASO USO B - ANALISI STATISTICA
% \begin{usecase}
% \addtitle{Caso d'uso B}{Analisi statistica} 
% \addfield{Scope:}{system-wide}
% \addfield{Goal:}{ottenere statistiche sulle prestazioni dell'algoritmo su tanti problemi UCARPP.}
% \addfield{Attore primario:}{i designer della metaeuristica.}
% \addfield{Precondizioni:}{
% 	\begin{itemize}
% 	  \item una directory del filesystem realmente esistente.
% 	  \item un intero $K$ tra i valori {2,3,4}.
% 	\end{itemize}}
% \addfield{Postcondizioni:}{un report statistico corretto 
% 	\footnote{una struttura dati contenente:
% 		\begin{itemize}
% 		  \item i due vettori degli scostamenti relativi dalle due benchmark
% 		  \item i due massimi dei suddetti vettori
% 		  \item i due minimi dei suddetti vettori
% 		  \item i due valori medi dei suddetti vettori
% 		  \item le due varianze dei suddetti vettori
% 		\end{itemize}},
% 		il cui formato è adattato a quello utilizzato nelle benchmark.}
% \addscenario{Scenario principale di successo:}{
% 	\item l'utente chiama il programma indicando la directory dove ricercare le istanze di problema e 
% 		  con quanti veicoli risolverle.
% 	\item su ciascun file presente nella directory viene eseguito il caso d'uso A. 
% 	\item il programma termina le analisi e restituisce un rapporto testuale di quanto eseguito.
% }
% \addscenario{Estensioni:}{
% 	\item[1.a] la sinopsi è invalida:
% 		\begin{enumerate}
% 		\item[1.] il sistema informa in modo generale dell'errore
% 		\item[2.] il programma termina senza successo.
% 		\end{enumerate}
% 	\item[2.a] almeno un file indicato non è presente:
% 		\begin{enumerate}
% 		\item[1.] il sistema informa del file non trovato
% 		\item[2.] il programma termina senza successo.
% 		\end{enumerate}
% 	\item[3.a] è implementato il modulo Matlab:
% 		\begin{enumerate}
% 		\item[1.] viene graficato un istogramma degli scostamenti relativi dalle benchmark.
% 		\end{enumerate}
% 	}
% \end{usecase}